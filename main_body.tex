\documentclass[11pt]{article}
\newcommand\newl{\vspace{3mm}}
\newcommand\tab{\hspace{4mm}}
\newcommand\formulabreak{\\\\}
\usepackage[margin=30mm]{geometry}
\usepackage{amsmath}
\title{\textbf{Fundamentals of Cryptographic Ciphers}}
\author{Chris Euerle, Eric Harrison, Claire Pitman, Kenneth Wayman}
\date{December 3rd, 2014}
\begin{document}
\maketitle
\section{Introduction}

\tab
In the modern world, the security of information is paramount. 
Every web developer has been told to 'salt-and-hash' user passwords at least once in their life, and we all know the storied lives of Alice and Bob and their attempts to communicate securely. 
Far too often, however, developers implementing large-scale systems either do not understand the origins of cryptography, leading to easily avoidable security vulnerabilities \cite{Software-Security}.

\newl

The relevance is especially apparent in light of recent attacks like that perpetrated against Sony this past November.  

\section{Related Work}%-------------------------------------------------------
\tab	
In recent years, there has not been much serious research work into ciphers where the two communicating parties need to meet as public-key cryptography has largely taken over.  
The reason for public-key's dominance is self-evident: plaintext communication over the internet is inherently insecure. 
Therefore, in order to use a secret key-only cipher, the two parties must have a way to securely communicate a secret key.  
But, if they have a way to securely transmit a secret key, why would they not just use that channel to communicate? \cite{Coursera}   
This limitation on key exchange is essentially insurmountable, and thus secret key cryptography is now more-or-less obsolete.  

\newl

Since purely secret-key-based ciphers are no longer the subject of significant research, it is perhaps more instructive to consider the contents of a basic cryptography course as \emph{related work}, as these now encompass the majority of existing research into such methods.
In a course by William Gasarch at the University of Maryland, College Park, for example, "pre-modern" methods, including shift ciphers, affine ciphers, and Vign\`ere ciphers are covered in detail \cite{Alice-and-Bob}.
A similar, though admittedly more perfunctory treatment of these ciphers is also given by Dan Boneh in the cryptography course he teaches for Stanford University online.
While there is nothing wrong with these treatments per se, it has been our experience that it is far more effective to learn, especially in Computer Science, by actually designing and implementing something in code.
\newpage
\section{Procedure}%----------------------------------------------------  -----
\tab	
To that end, we decided to implement some basic substitution ciphers in python. The three ciphers that we chose were the basic shift cipher, the affine cipher, and the Vign\`ere cipher.  These form a natural progression, from simplest to most complex, that builds off of the very simple concept of substitution.  We selected them because they are both simple and instructive enough to give insight into the origins of cryptography.
\subsection{Shift Cipher}
\tab
The shift cipher is one of the most well known types of cipher, likely due to the popularity of the \emph{Caesar cipher}.  It uses a single integer key between 0 and 25 both in encoding and decoding the message.  The basic encode algorithm for a shift cipher with key \emph{K} is as follows:

\begin{enumerate}
\item
For every character in the message text \emph{M}, assign the character a cardinal number corresponding to its position in the alphabet, e.g.
\begin{displaymath}
\{A \mapsto 0,\ B \mapsto 1, . . ., Z \mapsto 25\}
\end{displaymath}
Call that number \emph{X}.
\item
Use \emph{X} to calculate the cipher number \emph{N}:
\begin{displaymath}
N=( X + K ) mod 26
\end{displaymath}
\item
Map \emph{N} back to the character domain to obtain the ciphertext
\end{enumerate}
\tab
In order to communicate using a shift cipher, the two parties simply have to communicate their integer key, and the message can be decoded by reversing the method above.  Unfortunately, a shift cipher is extremely insecure, as the brute force method to crack a shift cipher is neither computationally expensive nor theoretically challenging: just try every possible shift. 
\subsection{Affine Cipher}

\subsection{Vign\`ere Cipher}
\section{Evaluation}

\section{Conclusion}


\bibliography{bibliography}{}
\bibliographystyle{plain}
\end{document}
