\documentclass[11pt]{article}
\usepackage[margin=30mm]{geometry}
\title{\textbf{Fundamentals of Cryptographic Ciphers}}
\author{Chris Euerle, Eric Harrison, Claire Pitman, Kenneth Wayman}
\date{}
\begin{document}

\maketitle

\section{Introduction}

\hspace{4mm} In our modern, computer-filled world, the security of information is paramount. Every web developer has been told to 'salt-and-hash' user passwords at least once in their life, and we all know the storied lives of Alice and Bob and their attempts to communicate securely.  Far too often, however, developers implementing large-scale systems either do not understand the origins of cryptography, leading to easily avoidable security vulnerabilities \cite{Software-Security}.

\vspace{3mm}

The relevance is especially apparent in light of recent attacks like that perpetrated against Sony this past November.  

\section{Related Work}
\hspace{4mm} 	In recent years, there has not been much serious research work into ciphers where the two communicating parties need to meet, as public-key cryptography has largely taken over.  The reason for public-key's dominance is self-evident: plaintext communication over the internet is inherently insecure, so it would be impossible for the two partes to establish a secure secret key.  Because of its limitations, cryptography that depends entirely on a secret key is more-or-less obsolete.  

\vspace{3mm}

	Because of the lack of modern research into purely secret key-based ciphers, it is perhaps more instructive to consider the contents of a basic cryptography course, as these now encompass the majority of past research into such methods.  
\section{Procedure?}
\hspace{4mm}	Our first step was to recreate the basic ciphers of pre-modern cryptography \cite{Alice-and-Bob}.  We first implemented a shift cipher, and from there implemented both an affine shift cipher and a Vign\`ere cipher.  
\section{Evaluation}

\section{Conclusion}


\bibliography{bibliography}{}
\bibliographystyle{plain}
\end{document}
